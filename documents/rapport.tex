\documentclass[a4paper,11pt]{report}
\usepackage[utf8]{inputenc}
\usepackage{color,amsmath,xcolor,listings,graphicx}
\usepackage[francais]{babel}

%% paramétrage pour les zones de 'code'
\lstset{
    language=Perl, commentstyle=\textit, frame=shadowbox,
    rulesepcolor=\color{gray}, basicstyle=\ttfamily\small, columns=flexible,
    tabsize=3, extendedchars=true, showspaces=false,
    showstringspaces=false, numbers=left, numberstyle=\tiny,
    breaklines=true, breakautoindent=true, captionpos=b, morecomment=[l]{//}
%language=Octave %-> choose the language of the code
%basicstyle=\footnotesize %-> the size of the fonts used for the code
%numbers=left %-> where to put the line-numbers
%numberstyle=\footnotesize %-> size of the fonts used for the line-numbers
%stepnumber=2 -> the step between two line-numbers.
%numbersep=5pt -> how far the line-numbers are from the code
%backgroundcolor=\color{white} -> sets background color (needs package)
%showspaces=false -> show spaces adding particular underscores
%showstringspaces=false -> underline spaces within strings
%showtabs=false -> show tabs within strings through particular underscores
%frame=single -> adds a frame around the code
%tabsize=2 -> sets default tab-size to 2 spaces
%captionpos=b -> sets the caption-position to bottom
%breaklines=true -> sets automatic line breaking
%breakatwhitespace=false -> automatic breaks happen at whitespace
%morecomment=[l]{//} -> displays comments in italics (language dependent)
}


%% infos du document
%%\title{Bureau d'étude 2014 - Télécommunication : Courant Porteur de Ligne \& Filtres}
\title{Courant Porteur de Ligne \& Filtres}
\author{Loïc Barbaresco, Rémi Barbaste, Robin Degironde, Émeric Tosi}
\date{\today}


\begin{document}

%% Afficher la page de garde : Titre + Auteur(s) + Date de dernière compilation
    \maketitle{}


%    \begin{figure} % on s'en fout de l'image moche xD
%        \begin{center}
            %\includegraphics{network.png}
            %\includegraphics[height=128, width=128]{network.png}
            %\includegraphics[scale=0.5]{network.png}
%        \end{center}
%            \caption{ Laule } % ce qui apparait juste en dessous de l'image
%            \label{c'est styler !}
%    \end{figure}

    \setcounter{tocdepth}{1} % définir la profondeur de l'Index
    \renewcommand{\contentsname}{Sommaire} % renommer l'Index en Sommaire
    \tableofcontents{} % afficher l'Index
    \clearpage


%% Différentes Parties / Chapitres / Autres fichiers à inclure :
    
\chapter{Courant Porteur de Ligne}
    \section{Introduction}
        \paragraph{}
lolilol !

    \clearpage
    
\chapter{Filtres}
    \section{Introduction}
        \paragraph{}
lolilol !

    \section{Code}
           \lstset{
                language=bash, basicstyle=\ttfamily\small, columns=flexible,
                tabsize=2, extendedchars=true, showspaces=false,
                showstringspaces=false, numbers=left, numberstyle=\tiny,
                breaklines=true, breakautoindent=true, captionpos=b
            }

    \begin{lstlisting}
    blablabla
    //...........................//
    blablabla
    \end{lstlisting}

    \clearpage

%% Annexes
    
\chapter{Annexes}
    \paragraph{}
lolilol !


    \clearpage

%% Index des images du rapport
    \listoffigures
    \clearpage

\end{document}
